\chapter{Conclusion}\label{ch6}

IoT devices often exchange real-time sensing and control data as small but numerous messages, requiring any backend service handling the huge amount of devices be extremely scalable and reliable to support large scale applications. To cope with the challenge, new paradigms such as the Fog Computing has been studied, which attempts to distribute possible centralized bottleneck. The new trend raises a new question besides the existing scalability problem, that is, to what extent could the same solution scales up and down while showing a consistent behaviour. 

In this thesis, in order to explore the possibility of building such scalable and reliable IoT services using concurrency-oriented language, an Erlang based CoAP server/client prototype called ecoap is presented. Following idiomatic design patterns, the ecoap architecture is divided to socket manager, endpoints and handlers, which can represent any number of real world concurrent activities. As a result, better scalability and reliability can be achieved in different environments. Processing chain of the protocol is separated in different components. Pattern matching of Erlang naturally fits with the binary encoded CoAP while message exchange states including deduplication and retransmission are implemented considering the dynamic nature of the language. APIs are provided in a straight forward manner and allows implementing customized functionality that exposes a RESTful interface to outside world without much effort. It is possible to send requests synchronously or asynchronously to CoAP endpoints and establish observe relations. The client component can also be combined with a server instance which enables complex business logic. Moreover, the consistent concurrency model of Erlang implies the business logic can be implemented and tuned to any desired concurrent environment.

To benchmark the prototype server, an Erlang CoAP benchmark tool was developed, which is the simplified version of the existing Java benchmark tool, CoAPBench. To have a clearer impression of how the server performs, it is compared with the mainstream CoAP implementation, Californium. Results in unconstrained cloud environment indicate that ecoap has comparable performance to Californium. In particular, ecoap scales smoothly up to 16 cores, delivering stable throughput and low latency even when 10,000 clients concurrently communicate with the server. ecoap also shows in general better response time. In contrast, Californium reaches limitations of scaling up during the evaluation, and suffers from inconsistent latency. It is inferred that the JVM's concurrency model and memory management might not cope with the constraints of CoAP very well. On the other hand, ecoap outperforms Californium when both run on Raspberry Pi, which is a constrained platform. ecoap has higher throughput and lower latency while Californium crashed under high load. It is considered that the problem of Californium which leads to its instability in cloud environment eventually made the server discontinued on a platform with too limited resources. Whereas the fair scheduling and process independent garbage collection ensures ecoap to be responsive and efficient in memory management in this case. In general, ecoap gives consistent behaviour in both environments and can automatically scale up and scale down. The flexibility of such a model reduces the effort of building scalable services. Moreover, the fault-tolerance test 
shows ecoap could keep service available even under high failure rate. It is considered as an evidence that the idea behind Erlang applies to IoT applications, which greatly eases the burden of achieving high reliability.

The IoT essentially requires more scalability and reliability since it shares similarity with the Web while being a more complex and dynamic system. The thesis demonstrates that the philosophy of Erlang does not only apply to large scale Web systems but also to the emerging IoT, and how combination of the two can lead to scalable and reliable IoT services. 


\subsubsection{Limitations and Future Work}

There are primarily four limitations of this work. First, the scope of this thesis is limited to vertical scaling, which means optimizing performance that can be achieved on a single machine. Though the results show the prototype could already handle a large amount of devices, horizontal scaling is necessary to further improve system capacity and reliability. It is of more practical value to continue to investigate this area, possibly based on the preliminary discussions in \autoref{horizontal_scale_note}. 

Second, the experiments in unconstrained environment is conducted using virtualized cloud service, which might have unexpected effect on certain applications. Although it is a common trend to deploy services in cloud and the experiments revealed performance in such a scenario, it might be more appropriate to study the behaviours in a fully controlled environment such as a dedicated physical machine. 

Third, fault-tolerant tests conducted in this work is still in its initial stage. That is to say, a more sophisticated scenario has to be developed for faults injection instead of fixed rate error. Furthermore, considering the first point, faults under a distributed environment might be of greater research interest. 

The last limitation is security aspects are not considered in this thesis, in particular, the DTLS layer. The extra layer might add undesired overhead to the system. Similar experiments need to be run with the secured version of CoAP against other implementations in further studies.








