\documentclass{uofsthesis-cs}

\usepackage{graphicx}

\usepackage{subcaption}

\usepackage{hyperref}

\usepackage{enumitem}

\usepackage{tabularx}

\usepackage{bbding}

\usepackage{makecell}

\DeclareRobustCommand\nocite[1]{%
����{\def\cite##1{\ignorespaces}#1}}
\newcommand\nocitecaption[1]{\caption[\nocite{#1}]{#1}}

% Documentation for the uofsthesis-cs class is given in uofsthesis-cs.dvi
% 
% It is recommended that you read the CGSR thesis preparation
% guidelines before proceeding.
% They can be found at http://www.usask.ca/cgsr/thesis/index.htm

%%%%%%%%%%%%%%%%%%%%%%%%%%%%%%%%%%%%%%%%%%%%%%%%%%%%%%%%%%%%%%%%%%%%%%%%%%%%%%
% FRONTMATTER - In this section, specify information to be used to
% typeset the thesis frontmatter.
%%%%%%%%%%%%%%%%%%%%%%%%%%%%%%%%%%%%%%%%%%%%%%%%%%%%%%%%%%%%%%%%%%%%%%%%%%%%%%

% THESIS TITLE
% Specify the title. Set the capitalization how you want it.
\title{CoAP infrastructure for IoT}

% AUTHOR'S NAME
% Your name goes here.
\author{Heng Shi}

% DEGREE SOUGHT.  
% Use \MSc or \PhD here
\degree{\MSc}         

% THESIS DEFENCE DATE
% Should be month/year, e.g. July 2004
\defencedate{December/2017}


% NAME OF ACADEMIC UNIT
%
% The following two commands allow you to specify the academic unit you belong to.
% This will appear on the title page as
% ``<academic unit> of <department>''.
% So if you are in the division of biomedical engineering you would need to do:
% \department{Biomedical Engineering}
% \academicunit{Division}
%
% The default is ``Department of Computer Science'' if these commands
% are not given.
%
% If you are in a discipline other than Computer Science, uncomment the following line and
% specify your discipline/department.  Default is 'Computer Science'.
% \department{If not Computer Science, put the name of your department here}

% If you are not in a department, but say, a division, uncomment the following line.
% \academicunit{Put the type of academic unit you belong to here, e.g. Division, College}


% PERMISSION TO USE ADDRESS
%
% If you are not in Comptuer Science you will want to change the
% address on the Permission to Use page.  This is done using the
% \ptuaddress{}.  Example:
%
% \ptuaddress{Head of the Department of Computer Science\\
% 176 Thorvaldson Building\\
% 110 Science Place\\
% University of Saskatchewan\\
% Saskatoon, Saskatchewan\\
% Canada\\
% S7N 5C9
% }

% ABSTRACT
\abstract{
This is the abstract of my thesis.  
}

% THESIS ACKNOWLEDGEMENTS -- This can be free-form.
\acknowledgements{
Acknowledgements go here.  Typically you would at least thank your supervisor.
}

% THESIS DEDICATION -- Also free-form.  If you don't want a dedication, comment out the following
% line.
%\dedication{This is the thesis dedication (optional)}

% LIST OF ABBREVIATIONS - Sample  
% If you don't want a list of abbreviations, comment the following 4 lines.
\loa{
\abbrev{CoAP}{Constrained Application Protocol}
\abbrev{CoRE}{Constrained RESTful environments}
\abbrev{CSP}{Communicating Sequential Processes}
\abbrev{DDS}{Data Distribution Service}
\abbrev{DETS}{Disk Erlang Term Storage}
\abbrev{DICE}{DTLS In Constrained Environments}
\abbrev{DTLS}{Datagram Transport Layer Security}
\abbrev{HTTP}{Hypertext Transfer Protocol}
\abbrev{IETF}{Internet Engineering Task Force}
\abbrev{IoT}{Internet of Things}
\abbrev{M2M}{machine-to-machine}
\abbrev{MQTT}{Message Queue Telemetry Transport}
\abbrev{NAT}{Network Address Translation}
\abbrev{REST}{Representative State Transfer}
\abbrev{OTP}{Open Telecom Platform}
\abbrev{RTPS}{Real-Time Publish-Subscribe}
\abbrev{TCP}{Transmission Control Protocol}
\abbrev{TLS}{Transport Layer Security}
\abbrev{UDP}{User Datagram Protocol}
%\abbrev{LOF}{List of Figures}
%\abbrev{LOT}{List of Tables}
}

%%%%%%%%%%%%%%%%%%%%%%%%%%%%%%%%%%%%%%%%%%%%%%%%%%%%%%%%%%%%%%%%
% END OF FRONTMATTER SECTION
%%%%%%%%%%%%%%%%%%%%%%%%%%%%%%%%%%%%%%%%%%%%%%%%%%%%%%%%%%%%%%%%

\begin{document}

% Typeset the title page
\maketitle

% Typeset the frontmatter.  
\frontmatter

%%%%%%%%%%%%%%%%%%%%%%%%%%%%%%%%%%%%%%%%%%%%%%%%%%%%%%%%%%%%%%%%
% FIRST CHAPTER OF THESIS BEGINS HERE
%%%%%%%%%%%%%%%%%%%%%%%%%%%%%%%%%%%%%%%%%%%%%%%%%%%%%%%%%%%%%%%%

\input intro.tex
\input problem_statement.tex
\input related_work.tex
\input architecture_and_implementation.tex
\input evaluation.tex
\input conclusion.tex

% Since thesis chapters are very long and there are a lot of them, it is recommended
% that you put each chapter in a separate .tex file and \input each one of them
% in order.  For example:
%
% \input chap1.tex
% \input chap2.tex
% ...
%
% The \input command inserts contents of the specified file at the point of the command.

%%%%%%%%%%%%%%%%%%%%%%%%%%%%%%%%%%%%%%%%%%%%%%%%%%%%%%%%%%%%%%%
% SUBSEQUENT CHAPTERS (or \input's)  GO HERE
%%%%%%%%%%%%%%%%%%%%%%%%%%%%%%%%%%%%%%%%%%%%%%%%%%%%%%%%%%%%%%%






%%%%%%%%%%%%%%%%%%%%%%%%%%%%%%%%%%%%%%%%%%%%%%%%%%%%%%%%%%%%%%%%
% The Bibliograpy should go here. BEFORE appendices!
%%%%%%%%%%%%%%%%%%%%%%%%%%%%%%%%%%%%%%%%%%%%%%%%%%%%%%%%%%%%%%%%


% Typeset the Bibliography.  The bibliography style used is "plain".
% Optionally, you can specify the bibliography style to use:
% \uofsbibliography[stylename]{yourbibfile}

\uofsbibliography[abbrv]{/Users/wilbur/Documents/Usask/PapersAndReferences/references.bib}

% If you are not using bibtex, comment the line above and uncomment
% the line below.  
%Follow the line below with a thebibliography environmentand bibitems.  
% Note: use of bibtex is usually the preferred method.

%\uofsbibliographynobibtex


%%%%%%%%%%%%%%%%%%%%%%%%%%%%%%%%%%%%%%%%%%%%%%%%%%%%%%%%%%%%%%%%%%%%%%%%%
% APPENDICES
%
% Any chapters appearing after the \appendix command get numbered with
% capital letters starting with appendix 'A'.
% New chapters from here on will be called 'Appendix A', 'Appendix B'
% as opposed to 'Chapter 1', 'Chapter 2', etc.
%%%%%%%%%%%%%%%%%%%%%%%%%%%%%%%%%%%%%%%%%%%%%%%%%%%%%%%%%%%%%%%%%%%%%%%%%%

% Activate thesis appendix mode.
\uofsappendix

% Put appendix chapters in the appendices environment so that they appear correcty
% in the table of contents.  You can use \input's here as well.
\begin{appendices}

\chapter{Sample Appendix}

Stuff for this appendix goes here.

\chapter{Another Sample Appendix}

Stuff for this appendix goes here.

\end{appendices}

\end{document}
