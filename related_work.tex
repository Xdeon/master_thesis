\chapter{Related Work}\label{ch3}

\section{Fog Computing}

In contrast to traditional cloud model, where many sensors upload raw data directly to a central cloud infrastructure for further processing and analysis, Cisco proposed fog computing concept which is a highly virtualized platform that provides compute, storage, and networking services between end devices and traditional cloud computing data centers, typically, but not exclusively located at the edge of network \cite{Bonomi:2012:FCR:2342509.2342513}. A Fog can consist of many heterogeneous, wide-spread geographically distributed sensor networks, which provide not only data collection, but also location awareness and rich service at the edge of the network, including applications with low-latency requirements such as real-time interactions and analytics, gaming and augmented reality. 
Applications could let local fog deal with machine-to-machine (M2M) interaction, such as collecting and processing data and issuing control commands to actuators, and let the rest of data to be consumed by higher tiers, which could be other fog or the global cloud for long-term analysis and storage. Typical scenarios of the fog includes connected vehicle, smart grid, and wireless sensor and actuator networks. 